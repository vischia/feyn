% feyn_test.tex
\documentclass[landscape]{article}
 \usepackage{feynmp}
 \usepackage[margin=0.2in]{geometry}
 \usepackage[pdftex]{graphicx}
\usepackage{subfig}
 \DeclareGraphicsRule{*}{mps}{*}{} 
 \begin{document}
 \begin{fmffile}{fgraphs2}
     \centering
     %(along, up)
     % Subfloat is necessary (with \begin{figure}) only if I have two graphs that I want to display side by side.
     %     \subfloat{
     \resizebox{2.5\width}{!}{
       \begin{fmfgraph*}(300,200)
         \fmfleft{i1,i2}
         \fmfright{o1,o2,o3}
         \fmf{gluon,label=$g$}{i1,w1}
         % l.side=left is necessary to force the label to be over of the line instead of below
         \fmf{gluon,label=$g$, l.side=left}{i2,w3}
         \fmf{fermion,label=$t$}{w3,w2}
         \fmf{fermion,label=$\bar{b}$}{w2,w1}
         \fmf{fermion,label=$\bar{t}$}{o3,w3}
         \fmf{boson,label=$H^{+}$}{w2,o2}
         \fmf{fermion,label=$b$}{w1,o1}
         %      \fmfv{lab=$V^{\ast}_{ud}$,lab.dist=0.05w}{w1}
       \end{fmfgraph*}
       %       }
     }
 \end{fmffile}
 \end{document} 
